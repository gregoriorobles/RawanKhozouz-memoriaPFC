\chapter{Resumen}
El porcentaje de alumnos solicitantes de las becas Erasmus aumenta considerablemente a medida que transcurren los a\~nos. Tanto, que en Internet podremos encontrar multitud de \textit{websites}, foros, blogs, etc. que nos ofrecen gran informaci\'on sobre dichas becas: nuevos conocimientos, c�mo formalizar preparativos y algunas pautas necesarias para resolver cualquier duda que surja antes de emprendernos en esta nueva experiencia.\\

Con tanta informaci�n en Internet, �por qu� no crear una web donde se recoja la informaci\'on propia de nuestra universidad? Una red donde se muestre lo m\'as novedoso de las experiencias vividas por los alumnos que hayan aprovechado las becas Erasmus, y que sea accesible para los futuros aspirantes a dichas becas.\\

De esta manera surge la idea de dise\~nar y desarrollar una aplicaci\'on web, exclusivamente para los alumnos de la ETSIT, donde podr\'an encontrar toda la informaci\'on necesaria para su pr\'oxima etapa Erasmus. El objetivo de la aplicaci\'on consistir\'a en dar todo el apoyo necesario que requiera un alumno de la mano de otros que hayan estado anteriormente en las distintas ciudades Erasmus Europeas o mundiales.\\

Para el desarrollo de este proyecto se han utilizado recursos de HTML5 y CSS3 para crear una web con contenidos atractivos, p\'aginas interesantes, lo m\'as din\'amica posible, y cuyo uso sea tan intuitivo como sencillo para el usuario.\\
