\chapter{Abstract}
The number of applicant students to Erasmus scholarship increases significantly as the years pass. That much so in the Internet we can find many websites, forums, blogs, etc. which provide us a great information about these scholarships: new knowledge, and guidelines to formalize arrangements necessary to resolve any doubt may arise before we start a new experience.\\

With so much information on the Internet, why don't we create a website where proper information of our university is collected? This can be a network to show the newest of the experiences of those students who have taken advantage of the Erasmus grants, and be accessible for future applicants to same kind of scholarships.\\

Thus, the idea of designing and developing a web application arises, exclusively for students ETSIT, where they may find all the necessary information during their Erasmus periods. The purpose of the application consists in providing all the necessary support to student from the hand of others who have previously been enrolled in different Erasmus European cities or any other sites.\\

For the development of this project some HTML5 and CSS3 resources have been used to create a website with attractive content, interesting pages being as dynamic as possible, and whose use is so intuitive and simple.\\
